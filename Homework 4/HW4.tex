\documentclass[12pt]{amsart}

\usepackage{epsf}
\usepackage{geometry}
\usepackage{listings}
\usepackage{algorithmic,algorithm}
\usepackage[notcite,notref]{showkeys}
\usepackage{multirow}
\usepackage{enumerate}

\usepackage[pdftex]{graphicx}
\usepackage{amscd}
\usepackage[pdftex]{color} % black,white,red,green,blue,cyan,magen ta,yellow
\usepackage[pdftex,colorlinks]{hyperref}
\usepackage{graphicx,pst-eps,epstopdf}
\usepackage{lscape}
\usepackage{indentfirst}
\usepackage{latexsym}
\usepackage{amsmath, amsfonts, amssymb,mathrsfs}
\usepackage{subfigure,pstricks,pst-node}
\usepackage{pst-eps,epstopdf}
\usepackage{verbatim}
%\usepackage{showkeys}

\newenvironment{plan}
{\bigskip\hrule\bigskip\centerline{\bf PLAN}\begin{quote}\tt}
{\end{quote}\bigskip\hrule\bigskip}

\hypersetup{
    bookmarks=true,         % show bookmarks bar?
    unicode=false,          % non-Latin characters in Acrobats bookmarks
    pdftoolbar=true,        % show Acrobats toolbar?
    pdfmenubar=true,        % show Acrobats menu?
    pdffitwindow=true,      % page fit to window when opened
    pdftitle={},    % title
    pdfauthor={Xiaozhe Hu},     % author
    pdfsubject={},   % subject of the document
    pdfnewwindow=true,      % links in new window
    pdfkeywords={}, % list of keywords
    colorlinks=true,       % false: boxed links; true: colored links
    linkcolor=red,          % color of internal links
    citecolor=blue,        % color of links to bibliography
    filecolor=magenta,      % color of file links
    urlcolor=cyan           % color of external links
}

\geometry{letterpaper, margin=1in}
\linespread{1.1}

\numberwithin{equation}{section}
\numberwithin{table}{section}
\numberwithin{figure}{section}
\numberwithin{algorithm}{section}

\everymath{\displaystyle}

\begin{document}

\title[]{Math 226, Homework 4, Due Dec. 11, 2015}
\author{Sheng Xu}
%\date{}

\maketitle


For problems that include programming, please include the code and all outputted figures and tables.  Please label these clearly and refer to them appropriately in your answers to the questions.

\


\begin{enumerate}

\item Derive an explicit linear 2-step method of order 3 and discuss its consistency, stability, and convergence.
\
Recall 2-step Explicit Method has the following form:\
\begin{align*}
y_{n+1}=-a_0y_n-a_1y_{n-1}+h(b_0f_n+b_1f_{n-1})
\end{align*}\
Now if we want it to be consistent under order 3, we have the following 4 equation corresponding to m=0,1,2,3:
\begin{align*}
1+a_0+a_1=0\\
1-a_1-b_0+b_1=0\\
1+a_1-2b_1=0\\
1-a_1-3b_1=0\\
\end{align*}\
When we solve the equation, we get:
\begin{align*}
a_0=4\\
a_1=-5\\
b_0=4\\
b_1=2\\
\end{align*}\
So the 2-step order 3 method we get is $y_{n+1}=-4y_n+5y_{n-1}+h(4f_n+2f_{n-1})$ and it is consistent.\
However with root condition, we know its characteristic function is $\xi^2+4\xi-5=0$, which means $\xi_1=1,\xi_2=-5$. Thus $|\xi_2|>1$, meaning it is not stable and convergent.
\

\item Derive the most general two-stage Runge-Kutta method of order 2.\\
We use Taylor expansion for $y_{n+1}$: (Note $f_n=f(t_n,y_n)$)
\begin{align*}
y_{n+1}&=y_n+hy^{'}_n+\frac{1}{2}h^2y^{''}_n+o(h^3)\\
&=y_n+hf_n+\frac{1}{2}h^2[\frac{\partial f}{\partial t}_n+\frac{\partial f}{\partial y}_nf_n]+o(h^3)\bigotimes
\end{align*}\\
If we consider the R-K method $y_{n+1}=y_n+h\omega_1K_1+h\omega_2K_2$ with $K_1=f_n$, $K_2=f(t_n+hc_2,f_n+h\alpha_{21}K_1)=f_n+hc_2\frac{\partial f}{\partial t}_n+h\alpha_{21}\frac{\partial f}{\partial y}f_n+o(h^3)$ and set $c_1=0$\\
\begin{align*}
y_{n+1}&=y_n+h\omega_1 f_n + h\omega_2 [f_n+hc_2\frac{\partial f}{\partial t}_n+h\alpha_{21}\frac{\partial f}{\partial y}f_n]+o(h^3)\\
&=y_n+h(\omega_1+\omega_2)f_n+h^2[\omega_2 c_2\frac{\partial f}{\partial t}_n+\omega_2 \alpha_{21}\frac{\partial f}{\partial y}_nf_n]+o(h^3)
\end{align*}\\
Let's set $\alpha_{21}=\alpha$ and compare the equation with $\bigotimes$ above. We will have:\\
\begin{align*}
&\omega_1+\omega_2=1\\
&\omega_2 c_2=\frac{1}{2}\\
&\omega_2 \alpha_{21}=\frac{1}{2}
\end{align*}\\
Thus,\\
\begin{align*}
&\omega_1=1-\frac{1}{2\alpha}\\
&\omega_2=\frac{1}{2\alpha}\\
&c_2=\alpha
\end{align*}\\
Therefore, we get:$y_{n+1}=y_n+h[(1-\frac{1}{2\alpha})K_1+\frac{1}{2\alpha}K_2]$, where $K_1=f_n=f(t_n,y_n)$, $K_2=f(t_n+\alpha h,y_n+\alpha h K_1)$\\
\

\item Implement the forward Euler method, backward Euler method, Crank-Nicolson method, and Runge-Kutta 4-stage (RK4) method and apply them to solve the following ODE problem, respectively.
$$
\begin{cases}
&y'(t) = 6y(t) - 6y(t)^2, \quad t \in (0, 20]\\
&y(0) = 0.5
\end{cases}
$$
Using $h=0.5,\ 0.25,\ 0.1$ to compute the approximate solution of each method. Plot the approximate solutions on the time interval $[0,20]$ and explain your observations from the numerical experiments.\\
From the observation we can conclude that:\\
\begin{center}
\bf{Table 3-1 Relationship between Method and Convergence under certain h}
\end{center}
\tabcolsep0.2in
\begin{tabular}{|c|c|c|c|c|}
\hline
Method&Forward Euler&Backward Euler&C-N&RK4\\
\hline
$h=0.50$&F&F&F&TF\\
\hline
$h=0.25$&T&F&T&T\\
\hline
$h=0.10$&T&T&T&T\\
\hline
\end{tabular}
\\	
T: converge to right solution F: do not converge TF:converge to wrong solution\\
Here we notice that for both implicit methods (Backward Euler and C-N)(Figure3.2 and 3.3), they do not converge. It seems to be a contradiction to what we learned in class. However, if we check the iteration, the iteration blows up. So we can conclude that the problem lies in iteration, not in method. Take Backward Euler as an Example: When we try to use iteration, the condition for contraction function is $\Delta y_k=y_{k+1}-y_k\epsilon [1-\frac{1}{6h},1+\frac{1}{6h}]$, when h becomes small enough, it will converge(Here, $y_k$ is the result of $k$th iteration). When h is bigger, $y_k$ may diverges or converge locally\\
For explicit method (RK4 and Forward Euler), they don't blow up when h=0.5 in the domain, but do not converge or do converge incorrectly. The reason should lies in their convergent condition, which means $h<C$ for some constant C.\\
For all methods, when h becomes smaller, the figure is smoother.\\
\end{enumerate}

\end{document}

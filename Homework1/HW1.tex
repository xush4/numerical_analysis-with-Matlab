\documentclass[12pt]{amsart}

\usepackage{epsf}
\usepackage{geometry}
\usepackage{listings} 
\usepackage{algorithmic,algorithm}
\usepackage[notcite,notref]{showkeys}
\usepackage{multirow}
\usepackage{enumerate}

\usepackage[pdftex]{graphicx}
\usepackage{amscd}
\usepackage[pdftex]{color} % black,white,red,green,blue,cyan,magen ta,yellow
\usepackage[pdftex,colorlinks]{hyperref}
\usepackage{graphicx,pst-eps,epstopdf}
\usepackage{lscape}
\usepackage{indentfirst}
\usepackage{latexsym}
\usepackage{amsmath, amsfonts, amssymb,mathrsfs}
\usepackage{subfigure,pstricks,pst-node}
\usepackage{pst-eps,epstopdf}
\usepackage{verbatim}
%\usepackage{showkeys}

\newenvironment{plan}
{\bigskip\hrule\bigskip\centerline{\bf PLAN}\begin{quote}\tt}
{\end{quote}\bigskip\hrule\bigskip}

\hypersetup{
    bookmarks=true,         % show bookmarks bar?
    unicode=false,          % non-Latin characters in Acrobats bookmarks
    pdftoolbar=true,        % show Acrobats toolbar?
    pdfmenubar=true,        % show Acrobats menu?
    pdffitwindow=true,      % page fit to window when opened
    pdftitle={},    % title
    pdfauthor={Xiaozhe Hu},     % author
    pdfsubject={},   % subject of the document
    pdfnewwindow=true,      % links in new window
    pdfkeywords={}, % list of keywords
    colorlinks=true,       % false: boxed links; true: colored links
    linkcolor=red,          % color of internal links
    citecolor=blue,        % color of links to bibliography
    filecolor=magenta,      % color of file links
    urlcolor=cyan           % color of external links
}

\geometry{letterpaper, margin=3.0cm}
\linespread{1.1}

\numberwithin{equation}{section} 
\numberwithin{table}{section}
\numberwithin{figure}{section}
\numberwithin{algorithm}{section}

\everymath{\displaystyle}

\begin{document}

\title[]{Math 226, Homework 1, Due Oct. 2, 2015}
%\author{}
%\date{}

\maketitle


For problems that include programming, please include the code and all outputted figures and tables.  Please label these clearly and refer to them appropriately in your answers to the questions.

\begin{enumerate}

\item Consider the Bernstein polynomials, for $f \in C(I)$, $I = [0,1]$, $n=1,2,\cdots$, 
$$
B_nf(x) = \sum_{k=0}^n f (\frac{k}{n}) 
\begin{pmatrix}
n \\
k
\end{pmatrix}
x^k(1-x)^{n-k}
$$
\begin{enumerate}
\item Show that the operator $B_n: C(I) \mapsto \mathcal{P}_n(I)$ is linear, where $\mathcal{P}_n(I)$ is the space of polynomial functions of degree at most $n$ on $I$.
\item Show that $B_n$ is a positive operator, i.e. if $f(x) \geq 0$ on $I$, then $B_nf(x)\geq 0$ on $I$.
\item Let $f_0 = 1$, $f_1 = x$, and $f_2 = x^2$, show that $B_nf_0 = f_0$, $B_nf_1 = f_1$, and $B_nf_2 = \frac{n-1}{n}f_2 + \frac{1}{n} f_1$.
\end{enumerate}

\

\item Prove the Chebyshev Alternation Theorem.

\

\item Derive the error formula for the Hermite interpolation $p_{2n+1}$ on $I$, i.e.
$$
f(x) - p_{2n+1}(x) =\frac{1}{(2n+2)!} f^{(2n+2)}(\xi) \left[(x-x_{0}) (x-x_{1}) \cdots (x-x_{n})\right]^2,
$$
where $\xi \in I$ is a point determined by the $x, x_0, \cdots, x_n$.

\

\item Let $I = [-1,1]$ and $\mathcal{P}_n(I)$ be the space of polynomial functions of degree at most $n$ on $I$. For any $q \in \mathcal{P}_n(I)$, show that $\| q \|_{\infty} \leq K(n) \| q \|_2$ with $K(n) = \frac{n+1}{\sqrt{2}}$.

\

\item Consider the function $f(x) = \frac{1}{1+x^2}$. Write codes to find the piecewise linear polynomial interpolation and clamped cubic spline to approximate $f(x)$ on $[-5,5]$ with equally distributed points of mesh size $h$.  Plot the approximations and observe how the errors change when the mesh size $h$ decreases. 

\

\item \textbf{(Optional)} If you are only allowed to use no more than $200$ points in $[-5,5]$ to approximate $f(x) = \frac{1}{1+x^2}$ by the piecewise linear polynomial interpolation, design a strategy to distribute the points and make the error in $L_{\infty}$-norm as small as possible. Implement and report your error.

\end{enumerate}

\end{document}